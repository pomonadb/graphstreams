\documentclass{article}

\usepackage{amsmath,amsthm,amssymb,tkz-graph,float}

\newtheorem{thm}{Theorem}
\newtheorem{lem}[thm]{Lemma}
\newtheorem{cor}[thm]{Corollary}
\newtheorem{alg}[thm]{Algorithm}

\theoremstyle{definition}
\newtheorem{defn}[thm]{Definition}

\numberwithin{thm}{subsection}

\begin{document}

\begin{center} \Large Shortest Path definitions\end{center}

\section{Preliminary Definitions}

Last week we talked about several potential definitions of `temporal shortest
path' that have different semantic value. We will consider these definitions
and address their usefulness by examining algorithms and attempting to prove
different facts.

\begin{defn}
  An \textbf{edge-varying temporal network} is a tuple $G \in V \times E$ where
  the set $V$ is the set of vertices, and $E \in V^2 \times T^2$, where $V$ is
  the set of vertices and $T = [0,\infty)$.  The first vertex represents the
  A \textbf{vertex-varying temporal network} is a tuple $G \in V \times E$ where
  the set $V$ is the set of vertices -- each labelled with a start and end time,
  and $E \in V^2$ , a pair of vertices.
  When the vertex pair within the edges is ordered, call the graph
  \textbf{directed}; otherwise call it \textbf{undirected}.
\end{defn}

Specifically we will analyze the effect of these definitions on the coauthorship
and citation networks.


\begin{defn}
  A \textbf{coauthorship network} is an undirected edge-varying temporal network
  in which each node represents an author, and the existence of an edge represents
  a collaboration over a period of time.
\end{defn}

\begin{defn}
  Edge persitence is the amount of time for which an edge is present. More
  formally, in a graph $(V,E)$ the persistence of an edge $(u,v,t_1,t_2)$ is
  defined as $t_2 - t_1$.
\end{defn}

\begin{defn}
  A citation network is a directed edge-varying temporal network $(V,E)$ with
  infinite edge persistence. So for all edges $(u,v,t_1,t_2)\in E$,
  $t_2 = \infty$. For simplicity, we can write edges as $u \to_{t_1} v$ or
  $(u,v)_{t_1}$.  In this network, each node represents an author, and the
  existence of an edge $u \to_{t_1} v$ means that author $v$ cited author $u$
  at time $t$. This way the direction of the arrow represents the flow of
  information.
\end{defn}


Now we will move on to consider the different analysis methodologies that will
result from different temporal definitions of `shortest path'.

\begin{defn}
  A graph is a \textbf{path} if it is a simple graph whose vertices can be linearly
  ordered such that there is an edge $uv$ if an only if $u$ and $v$ are adjacent
  in the ordering.
  A digraph is a \textbf{path} if it is a simple directed graph whose vertices
  can be linearly ordered such that there is an edge $u \to v$ if and only if
  $v$ immediately follows $u$ in the ordering.
\end{defn}

\begin{defn}
 A \textbf{shortest path} between $v_1, u$ (di)graph $G = (V,E)$ is
 $P = v_1, v_2, \cdots, v_n, u$ such that $v_i \in V$ for all $i in [n]$,
 and $v_jv_{j+1}, v_nu \in E$, for all $j \in [n-1]$, and there is no path
 $P' = v_1, v_2, \cdots, v_m, u$ such that $m < n$.
 \end{defn}

Note that this definition does not enforce uniqueness of shortest paths.

\section{Consecutive Contemporaneity}

Here, we consider the addition of paths to include a temporal component, where
any two consecutive edges must share some contemporary period. This is a sensible
definition as two edges should not be able to form a path in a temporal network
if they did not happen at the same time. This is formally defined below.

\begin{defn}
  A \textbf{consecutive temporal path} between $v_1$ and $v_n$ is a path
  $P = v_1, v_2, \cdots, v_n$ such that $v_i \in V$ for all $i \in [n]$,
  and $v_jv_{j+1} \in E$, for all $j \in [n-1]$, and for every pair of
  consectutive edges $(v_{i-1},v_{i})_{t_1}$ and $(v_{i}, v_{i+1})_{t_2}$,
  $t_1 \leq t_2$.
\end{defn}

We will consider the consequenses of this definition inthe context of different
edge-behaviours, infinite persistence, and windowed persistence.

\subsection{Infinitely Persistent Edges}

The first model we will consider is the simplest of the three, where we disallow
edge-deletion. We will define the persistent coauthorship network to have this
property. [note about semantic equivalence to railway network?]

\begin{defn}
  The \textbf{persistent coauthorship network} a couthorship network $G = (V,E)$,
  where for all $(u,v,t_1,t_2) \in E$, $t_2 = \infty$. For simplicity, we can
  denote an edge by $(u,v)_{t_1}$ or $u -_{t_1} v$. Since this network is
  undirected, $(u,v)_{t_1} = (v,u)_{t_1}$.
\end{defn}

Then, we can consider what a reasonable definition of `shortest path' might be.
In this model, once an edge exists, it is always traversible, so if author
$a_1$ wrote a paper with author $a_2$ in 1932, and author $a_2$ wrote a paper with
$a_3$ in 1990, we can find a path between $a_1$ and $a_3$.  It is also feasible
that $a_1$ wrote a paper with $a_4$ in 1932 as well, and then $a_4$ and $a_3$
wrote a paper in 1933. These two paths $P_1 = a_1 -_{1932} a_2 -_{1990} a_3$, and
$P_2 = a_1 -_{1932} a_4 -_{1933} a_5$ should have some manner of differentation,
since the difference in start times of the edges is 1 58 in $P_1$ and only 1
in $P_1$. This distinction motivates a difference in fastest vs. shortest path.


\begin{defn}
  The \textbf{persistent shortest path} beteween $u$ and $v$ is a consecutive
  temporal path $u,v_1,\cdots,v_n,v$ such there exists no other $u,u_1, \cdots u_m,v$ such that $m < n$.

  The \textbf{persistent fastest path} bewteen $v_1$ and $v_n$ is consecutive
  temporal path $v_1,v_2,\cdots,v_n$, with first edge $(v_1,v_2)_{t_1}$ and last
  edge $(v_{n-1},v_n)_{t_{n-1}}$, such that there exists no other
  $u_1,u_2, \cdots, u_m$ with first edge $(u_1,u_2)_{s_1}$, last edge
  $(u_{m-1},u_m)_{s_{m-1}}$ and $s_{m-1} - s_1 < t_{n-1} - t_{1}$.
\end{defn}


\begin{cor}
  Shortest path in persistent coauthorship network is the same as the shortest
  path in the aggregated static graph.
\end{cor}

\begin{proof}[Proof. (Idea)]
  Since the edges have infinite persistence, can just wait at a vertex until
  the desired edge in the aggregated graph shows up.
\end{proof}


\subsection{Windowed edges}

Now consider that we in fact limit the persistence of the edges with an endpoint
specific to each edge (as is specified in the definition of an edge-varying
temporal graph). The definition of this graph is the same as in definition 1.1
and 1.2. We call this graph a \textbf{windowed coauthorship network} or simply
a \textbf{coauthorship network}. If we specify a universal edge-persistence $t$
such that for all edges $(u,v,t_1,t_2)$ in the network, $t = t_2 - t_1$, then
we call this coauthorship network \textbf{$t$-windowed}.

When we consider the above definition (2.1) of a path it is clearly too
simplistic, as it does not consider that an edge may cease to exist. So let
us consider a new definition of a temporal path.

\begin{defn}
  A \textbf{windowed temporal path} bewteen $v_1$ and $v_n$ is a collection of
  edges $(v_1,v_2,s_1,t_1),(v_2,v_3,s_2,t_2), \cdots (v_{n-1},v_{n}, s_{n-1}, t_{n-1})$
  such that $v_1$ and $v_n$ have degree $n$, and the remaining vertices have
  degree 2.  Most importantly, $t_i \geq s_{i+1}$ for all $i \in [n-2]$.
\end{defn}

The definitions for fastest and shortest path will be the same as in definition
2.1.2.


\section{Pairwise contemporaneity}

Here we can consider many of the same definitions, but under are different
lens of contemporaneity for paths. Here we want all edges to have some overlap
in their time interval.

\begin{defn}
  A \textbf{pairwise contemporary temporal path} between $v_1$ and $v_n$ is a
  collection of edges $(v_1,v_2,s_1,t_1), (v_2,v_3,s_2,t_2), \cdots (v_{n-1},v_n,s_{n-1}, t_{n-1})$, such that $\bigcap_{i \in [n-1]} [s_i, t_i] \neq \emptyset$.
\end{defn}

As might be expected, pairwise contemporary temporal paths behave the same way
that consecutive temporal paths do in the persistent coauthorship network.
Since the edges have infinite persistence, all edges are contemporary `at
infinity.'

In the case of the windowed coauthorship network, the definitions remain the same
for shortest and fastest paths.


\end{document}