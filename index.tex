\section{Temporal Index}

There are many efficient indexing techniques. But the temporal dimension allows
us an additional method by which we can filter our data. In fact it is a very
strong filter that can prune a search space from tens of millions of edges to on
average about 10000 per arbitrary time slice. So how do we go about constructing
a temporal index?  We start with the observation that each edge is a tuple in
$V^2 \times \T^2$. Similar to how adjacency matrices model the
edge-relationships in the 2 dimensional space $V^2$, we can create the
2-dimensional space $T^2$, where each time window represents a region in this
space. For any given edge $e = (u,v, t_s, t_e)$, we will create a 2-dimensional
box $b(e)$ bounded by the system
\[b((u,v, t_s, t_e)) =
  \begin{cases}
    x \leq t_e \\
    x \geq t_s \\
    y \leq t_e \\
    y \geq t_S
  \end{cases}
\]

Notice that $T(e) \cap T(f) \neq \emptyset$ for $e,f$ edges, if and only if
$b(e)$ and $b(f)$ overlap. We can then use an R+Tree to index this space. In
this form, this tool is only useful for when we consider $c = \texttt{concur},
\texttt{Sconsec}$.  More robustly we can create a specific index for when $c =
\texttt{Wconsec}$, in which case we will generate the 2 dimensional space
\[ \begin{cases}
     x \leq t_e \\
     y \leq t_e
   \end{cases}
\]

Thus, like everything else in this paper, our indexing methodology is dependent
on the contemporaneity condition $c \in \C$. 

Also, note that this index can be used for much more than just representing the
edges in the graphs, it can also be used for indexing subgraphs based on the output
of $\tau_f(c)$.  Then, for any edge that we are curious about adding, we can use
the index to figure out which edges statisfy the temporal condition. \todo{Read
  and cite the R+Tree paper}

  
